\chapter{Реализация методов}
Предложенные в данной работе методы реализованы в системе верификации Klever для проверки требований к программным системам на языке программирования Си с расширениями GNU методом верификации моделей программ~\cite{klever}.
Klever является проектом с открытым исходным кодом, все компоненты которого реализованы на языке программирования Python 3.

Система Klever может быть установлена и использована на различных дистрибутивах операционной системы Linux и автоматически развернута и использована в облачной IaaS платформе OpenStack.

Для подготовки баз сборки используется инструмент Clade~\cite{clade}.
Clade выполняет контролируемую сборку программы на языке программирования Си для перехвата команд и извлечения необходимой информации о программном интерфейсе.
Для получения данных о файлах, функциях, глобальных переменных, типах и макросах используется механизм запросов к исходному коду, реализованный на основе упомянутого ранее инструмента CIF.

Структура верификационной системы соответствует архитектуре, предложенной в главе 3 данной работы.
Далее рассмотрена реализация отдельных компонентов системы верификации Klever.

\section{Сервер}
Сервер разработан на основе веб-фреймворка Django.
Для хранения компонент использует базу данных под управлением PostgreSQL.
Для установки пользователи могут использовать веб-серверы Apache2 с mod\_wsgi или NGINX с Gunicorn.

Пользовательский интерфейс сервера реализует все функции, необходимые для работы пользователей, которые были обозначены в третьей главе.
Работу с интерфейсом могут выполнять несколько пользователей с различными ролями, для которых настраиваются подходящие представления информации и права доступа.

Для анализа результатов пользователем реализованы представления ошибочных путей и отчетов о покрытии.
Разработаны вспомогательные промежуточные форматы, отличные от предложенных сообществом SV"~COMP, для добавления дополнительной информации к свидетельствам корректности и отчетам о покрытии в процессе первичной обработки результатов.
Интерфейс сервера позволяет сохранять подготовленные верификационные задачи в формате сообщества SV"~COMP для отладки инструментов верификации моделей программ и валидации результатов.

Для каждого свидетельства корректности или нарушения пользователь может указать экспертную оценку и отметить ряд атрибутов, добавленных генератором верификационных задач, по которым данное свидетельство может быть сопоставлено с другими при сравнении результатов решений верификационных заданий.
Атрибутами являются требование, имя проверяемой программы, ее версия и имя модуля.
При создании экспертной оценки пользователь отмечает, является ли вердикт решения верификационной задачи истинным или ложным и может оставить свой комментарий.
Экспертные оценки переносятся автоматически между верификационными задачами с одинаковыми значениями атрибутов.
Чтобы избежать ошибок при экспертизе из-за несовершенных алгоритмов переноса экспертных оценок и сравнения ошибочных путей, перенесенные оценки показываются в виде предложений пользователю с возможностью подтверждения или опровержения такой автоматической разметки результатов.

Для решаемых верификационных заданий оценивается прогресс решения на основе данных, периодически получаемых от генератора верификационных задач.
Оценка прогресса выполняется на основе подсчета числа решенных верификационных задач за равные интервалы времени и их общего числа.
Генератор верификационных задач в рамках каждого этапа своей работы периодически отправляет отчеты о ходе решения верификационного задания на сервер.
В зависимости от уровня отладки, состав отчетов может содержать различную информацию: журналы о выполнении компонентов, промежуточные данные, сообщения об ошибках, статистику о затраченных вычислительных ресурсах, отчеты о покрытии и свидетельства нарушения ошибок.
Вся полученная в данных отчетах информация становится сразу доступна пользователю для анализа и экспертизы.

Классификация ошибок и сбоев в инструментах верификации и компонентах генератора верификационных задач выполняется на основе \textit{описания проблем}.
Описание проблемы представляет собой идентификатор ошибки, описание ошибки на естественном языке и шаблон, заданный регулярным выражением, по которому ошибка обнаруживается в журнале выполнения компонента системы верификации.
При получении отчетов от генератора верификационных задач все журналы проверяются на предмет наличия проблем, описанных для соответствующих компонентов.
Система уже поставляется с набором описаний распространенных проблем, подготовленных заранее, что позволяет на раннем этапе диагностировать ошибки, допущенные пользователем при разработке спецификаций и конфигурировании системы верификации.

В процессе верификации моделей программ накапливаются различные данные, полученные автоматически системой верификации или подготовленные пользователями. 
К первому типу относятся свидетельства корректности и нарушений, база сборки, верификационные задачи, отчеты о покрытии, журналы выполнения компонентов и т.п.
Спецификации, экспертные оценки и описания проблем составляют данные, которые система верификации получает в процессе работы пользователя.
Некоторые результаты может требоваться хранить долгое время.
Поэтому сервер предлагает несколько режимов хранения результатов: полноценный и облегченный.
В полноценном режиме для каждого верификационного задания хранятся все артефакты, которые могут быть проанализированы пользователем.
После завершения работы пользователя с определенным верификационным заданием может быть включен облегченный режим хранения, в котором для данного задания удаляется большая часть данных, подготовленных автоматически системой верификации.

\section{Генератор верификационных задач}
Генератор верификационных задач начинает решение верификационного задания с получения конфигурационных параметров, базы сборки и спецификаций от сервера.
Затем выполняется проверка корректности и целостности полученных данных, включая разработанных пользователем спецификаций.
В случае, если все данные корректны запускаются следующие компоненты генератора:
\begin{itemize}
    \item компонент декомпозиции;
    \item компонент генерации верификационных задач;
    \item компонент обработки результатов;
    \item компонент обмена данными.
\end{itemize}

Затем выполняется декомпозиция программы на модули согласно методу, изложенному во второй главе.
Компонент выполняет шаги декомпозиции и предоставляет инфраструктуру для реализации различных стратегий выделения модулей и агрегации.
В рамках Klever уже реализованы стратегии выделения модулей для ядра ОС Linux и проекта BusyBox.

Для проверки требований к компонентам ядра ОС Linux была реализована стратегия выделения драйверов и подсистем.
Стратегия опирается на граф команд сборки и на основе специальных команд сборки для создания объекта ядра (англ. kernel object) драйвера, определяет входящие в его состав файлы на языке Си, а остальные файлы с исходным кодом из соответствующих директорий, компилируемые статически в ядро ОС, рассматриваются как подсистемы.
В результате стратегия выделяет модули, каждый из которых содержит исходный код либо драйвера, либо подсистемы.

Проект BusyBox объединяет несколько сотен пользовательских программ в виде одной программной системы.
Каждая программа называется апплетом (англ. applet).
Апплет состоит из нескольких файлов на языке программирования Си и опирается только на общую для проекта библиотеку функций libbb и стандартную библиотеку языка программирования Си.
Каждый апплет имеет строго одну точку входа, название которой строится как имя апплета и суффикс \textit{\_main}.
Сигнатура таких функций совпадает с сигнатурой функции \textit{main} пользовательских программ на языке программирования Си.
Стратегия для проекта BusyBox в качестве отдельных модулей выделяет библиотеку libbb и непосредственно апплеты.
В основе реализованной стратегии лежит алгоритм обхода графов файлов и функций для определения состава каждого апплета.

Для агрегации реализованы две стратегии на основе обхода графов функций и модулей с учетом отчетов о покрытии при верификации отдельных модулей.
Первая стратегия выполняет поиск в ширину модулей с определениями функций, отсутствующими в целевом модуле.
Вторая стратегия нацелена на поиск минимального набора модулей, который позволит вызвать как можно больше точек входа целевого модуля.
Отчеты о покрытии используются во второй стратегии для определения тех функций, вызывающих точки входа целевого модуля, которые достижимы при верификации с использованием некоторого набора спецификаций предположений об окружении.

Компонент генерации верификационных задач получает от компонента декомпозиции набор модулей и затем начинает параллельную подготовку для них моделей окружения и требований.
Процесс генерации отдельной верификационной задачи осуществляется набором плагинов, которые выполняют преобразования над исходным кодом модуля.
В конфигурации системы верификации для заданной программы пользователь выбирает те плагины, которые необходимы для подготовки соответствующих верификационных задач.
Первым плагином, как правило, запускается генератор моделей окружения, который в свою очередь содержит несколько построителей моделей сценариев и транслятор.

Для моделирования окружения драйверов и подсистем ядра ОС Linux разработаны два построителя: построитель моделей сценариев вызова обработчиков и построитель моделей сценариев вызова функций инициализации и выхода.
Генератор для вызова обработчиков требует от пользователя разработки спецификации предположений об окружении, описывающих сценарии вызова обработчиков определенных типов.
Формат спецификаций основан на формате задания промежуточной модели окружения, но содержит ряд дополнительных расширений.
Построитель моделей сценариев вызова функций инициализации и выхода требует для работы список макросов, при помощи которых можно определить функции инициализации и выхода драйверов и функции инициализации подсистем.
Набор таких макросов отличается от версии к версии ядра ОС Linux.
Модели сценариев, подготовленные построителем, выполняют вызов функций инициализации и выхода драйверов в соответствии с возможным порядком их загрузки и выгрузки операционной системой, который определяется при решении задачи топологической сортировки графа зависимостей между модулями.
Один из построителей моделей сценариев предназначен для вызова точек входа в произвольном порядке по списку имен функций или регулярному выражению, заданному пользователем.
Данный построитель может использоваться при верификации различных программ, включая апплеты BusyBox.

Транслятор генератора моделей окружения поддерживает подготовку параллельной модели окружения на языке программирования Си с использованием интерфейса управления потоками согласно стандарту POSIX.
Транслятор позволяет подготовить и последовательную модель окружения, но она является неполной из-за сокращения частичных порядков последовательностей событий разных моделей сценариев взаимодействия.
Практические эксперименты показали, что полная последовательная модель окружения, зачастую вызывает взрыв числа состояний в модели, построенной инструментом верификации, и ведет к ухудшению результатов верификации в целом.
Транслятор также позволяет использовать несколько разных наборов реализаций служебных функций для адаптации процесса синтеза моделей окружения для проверки разных требований и использования различных инструментов верификации моделей программ.

Для генерации моделей требований и компоновки исходного кода верификационных задач был реализован подход, апробированный в системе верификации LDV~Tools.
Необходимые компоненты были реализованы в виде плагинов.
Для инструментации и препроцессирования используется CIF, а для задания моделей окружения и требований соответствующее аспектно-ориентированное расширение языка Си.
CIF основан на GCC версии 7.1, поэтому инструмент поддерживает язык Си с расширениями GNU.
Компоновщик верификационных задач слайсера CIL из набора инструментов по дедуктивной верификации AstraVer Toolset~\cite{CILAstra}.
Формат генерируемых верификационных задач полностью следует формату SV"~COMP, поэтому в системе Klever могут применяться различные инструменты верификации.
Для задания конфигурационных параметров был предложен формат спецификаций, позволяющий описывать конфигурационные параметры для определенных версий и инструментов верификации в зависимости от проверяемого требования.

\section{Решатель верификационных задач и заданий}
Решатель верификационных задач и заданий предназначен для запуска экземпляров компонентов генератора верфикационных задач и инструментов верификации моделей программ.

Компонент осуществляет контроль за доступностью и потреблением вычислительных ресурсов и гарантирует изолированную работу данных компонентов.
Данная функциональность реализована при помощи BenchExec~\cite{Beyer2015}.

В рамках решателя были реализованы модуль прогнозирования ограничения на максимальный объем оперативной памяти, выполняющий расчет на основе ряда эвристических предположений и статистики измерения затраченных вычислительных ресурсов при решении верификационных задач.
Модуль планирования реализует жадный алгоритм выбора вычислительного узла для запуска экземпляров генератора верификационных задач и инструментов верификации с учетом приоритета верификационных заданий.
Основные модули запуска компонентов системы верификации позволяют решать верификационные задачи на одном вычислительном узле и в вычислительном кластере, управляемом при помощи VerifierCloud~\cite{VerifierCloud}.
Были также разработаны прототипы модулей запуска для использования систем управления вычислительным кластерами Docker Swarm~\cite{swarm} и Kubernetes~\cite{kubernetes}.

Для решения верификационных задач используются инструменты верификации моделей программ CPAchecker и Ultimate Automizer~\cite{Heizmann2015}.
Для интеграции других инструментов, поддерживающих формат верификационных задач сообщества SV"~COMP, требуется:
\begin{enumerate}
\item Описать конфигурационные параметры для требований, которые могут быть проверены при помощи данных инструментов.
\item Указать пути к исполняемым файлам инструментов в конфигурационных параметрах скриптов установки системы верификации Klever.
\end{enumerate}

На практике в системе верификации используются преимущественно различные версии инструмента верификации CPAchecker, в котором реализована выдача свидетельств нарушения и отчетов о покрытии в наиболее детальном виде.
Инструмент Ultimate Automizer, как и другие инструменты верификации сообщества SV"~COMP, позволяет получить результаты верификации, но из-за неполной информации об ошибочных путях анализ предупреждений об ошибках может быть трудным.

